%	!Mode::"UTF-8"
%	本模板设置基于北京大学交叉学院王宇哲学长的分享,特此感谢!
%	模板制作:北京大学化学与分子工程学院 王梓涵

%	仅限交流使用,不可用作商业用途

\documentclass[12pt]{article}

%	页面设置
\usepackage{geometry}
\geometry{left=2.5cm, right=2.5cm, top=2.5cm, bottom=2.5cm}
\usepackage{graphicx}
\usepackage{ctex}
\usepackage{fontspec}
\usepackage{setspace}
\usepackage[usenames,dvipsnames]{xcolor}
\usepackage{titlesec}

%	字体设置
\setmainfont{Times New Roman}
\setCJKmainfont{SimSun}
\setCJKsansfont{SimHei}

%	表格设置\
\usepackage{array,colortbl}
\usepackage{makecell}
\newcommand{\addcell}[2][4]{\makecell{\zihao{#1}\textsf{#2}}}
\usepackage{titlesec}
\usepackage{booktabs}
\usepackage{tabularx}

%	设置图注、表注
\usepackage{caption}
\usepackage{bicaption}
\captionsetup{labelsep=quad, font={small, bf}, skip=2pt}
\DeclareCaptionOption{english}[]{
    \renewcommand\figurename{Fig.}
    \renewcommand\tablename{Table}
}
\captionsetup[bi-second]{english}

%	设置页眉
\usepackage{fancyhdr}
\usepackage{xpatch}
\pagestyle{fancy}
\fancypagestyle{preContent}{
    	\fancyhead[L]{\zihao{-5} 物理化学实验}
    	\fancyhead[C]{\zihao{-5} 实验十\ \ 磁化率的测定}
    	\fancyhead[R]{\zihao{-5} 2100011837\ 王梓涵}
		\renewcommand{\headrulewidth}{2pt}
		\renewcommand{\footrulewidth}{1pt}
		\xpretocmd\headrule{\color{BrickRed}}{}{\PatchFailed} % 设置页眉分割线颜色
		\xpretocmd\footrule{\color{BrickRed}}{}{\PatchFailed} % 设置页脚分割线颜色
}
\pagestyle{preContent}



%	设置首页页眉及取消首页页脚 若不需要首页页眉 请注释掉下列内容
\fancypagestyle{plain}{
	\fancyhead[L]{\zihao{-5} 物理化学实验}
    \fancyhead[C]{\zihao{-5} 实验十\ \ 磁化率的测定}
	\fancyhead[R]{\zihao{-5} 2100011837\ 王梓涵}
	\cfoot{}
}

%	设置标题格式
\titleformat*{\section}{\color{BrickRed}\zihao{4}\sffamily}
\titleformat*{\subsection}{\zihao{-4}\sffamily}
\titleformat*{\subsubsection}{\zihao{-4}\sffamily}
\titlespacing*{\section}{0pt}{10pt}{10pt}
\titlespacing*{\subsection}{0pt}{10pt}{5pt}
\titlespacing*{\subsubsection}{0pt}{10pt}{5pt}


%	设置引用格式(ACS格式规范)
%	注意:请安装JabRef
%	JabRef使用参考:https://blog.csdn.net/weixin_44191286/article/details/85698921
\usepackage[super,round,comma,compress]{natbib}

%	数学公式增强
\usepackage{amsmath}
\usepackage{amssymb}

%	单位与数学式
\usepackage{siunitx}

%	设置封面
\begin{document}
    % 标题页
    \begin{titlepage}
    	% 页眉
    	\thispagestyle{plain}
        % 校徽图片
        \begin{figure}[h]
            \centering
            \includegraphics{pku.png}
        \end{figure}
        \vspace{24pt}
        % 标题
        \centerline{\zihao{-0} \textsf{\textcolor{BrickRed}{物理化学实验报告}}}
        \vspace{40pt} % 空行
        \begin{center}
            \begin{tabular}{cp{14.1cm}}
                % 题目
                \addcell[2]{实验十:} & \addcell[1]{磁化率的测定} \\
                \cline{2-2}
            \end{tabular}
        \end{center}
        \vspace{20pt} % 空行
        \begin{center}
            \doublespacing
            \begin{tabular}{cp{5cm}}
                % 姓名
                \addcell{\textcolor{BrickRed}{姓\phantom{空格}名:\ }} & \addcell{王梓涵} \\
                \cline{2-2}
                % 学号
                \addcell{\textcolor{BrickRed}{学\phantom{空格}号:\ }} & \addcell{2100011837}\\
                \cline{2-2}
                % 组别
                \addcell{\textcolor{BrickRed}{组\phantom{空格}别:\ }} & \addcell{22组} \\
                \cline{2-2}
                % 实验日期
                \addcell{\textcolor{BrickRed}{实验日期:\ }} & \addcell{2023.9.21}\\
                \cline{2-2}
                % 室温
                \addcell{\textcolor{BrickRed}{室\phantom{空格}温:\ }} & \addcell{301.75\ K}\\
                \cline{2-2}
                % 大气压强
                \addcell{\textcolor{BrickRed}{大气压强:\ }} & \addcell{100.81\ kPa}\\
                \cline{2-2}
            \end{tabular}
            \begin{tabular*}{\textwidth}{c}
                \\ % 这是空行
                \\ % 这是空行
                \\ % 这是空行
                \hline % 分割线
            \end{tabular*}
        \end{center}
        % 摘要
        \textsf{\textcolor{BrickRed}{摘\ \ 要}}\ \ 本次实验以摩尔盐为标准样,在约\qty{25}{\degreeCelsius}通过Guoy天平分别测量了$CuSO_{4}·5H_{2}O$、$K_{4}Fe(CN)_{6}·3H_{2}O$
		的摩尔比磁化率以及未知样品的磁化率。通过计算得到了$CuSO_{4}·5H_{2}O$、$K_{4}Fe(CN)_{6}·3H_{2}O$的摩尔比磁化率分别为$(2.376\pm a)\times 10^{-8} kg/mol$,$(2.376\pm b)\times 10^{-8} kg/mol$,约含有1个和0个单电子。
		未知样的比磁化率为$(2.376\pm a)\times 10^{-8} kg/m^{3}$。
        \\
        \\
        % 关键字
        \textsf{\textcolor{BrickRed}{关键词}}\ \ 磁化率;Guoy天平;摩尔磁化率;摩尔盐
    \end{titlepage}

    \section{引言}
	引言部分...
               
	\vbox{} % 设置空行
	     
    \section{实验部分}
    	\subsection{仪器和试剂}
    		仪器和试剂...
    			
    	 \subsection{实验内容}
			\subsubsection{实验内容1}
				实验内容1...
			\subsubsection{实验内容2}
				实验内容2...
			\subsubsection{实验内容3}
				实验内容3...
			\subsubsection{实验内容4}
				实验内容4...
			\subsubsection{实验内容5}
				实验内容5...
    	
	\vbox{}  
	
	 \section{数据与结果}
 		\subsection{实验数据记录及处理}
 			\subsubsection{实验内容1}
 				实验内容1...
		 		% 插入图片示例
		 		 \begin{figure}[h]
		 			\centering
		 			\includegraphics[width=0.5\textwidth]{pku.png}
		 			\bicaption{中文图题}{Caption}
		 		\end{figure}
		 	
		 		xxx如\textbf{表1}所示。
		 		
		 		% 插入表格示例
		 		  \begin{table}[h]
					\arrayrulecolor{BrickRed}
		 			\centering
		 			\zihao{5}
		 			\bicaption{中文表题}{Caption}
		 			\begin{tabular}{ccccc} 
		 				% 五列表格为例 改变列数请改变"c"的个数
		 				\toprule
		 				一 & 二 & 三 & 四 & \thead[c] {$E_{trans}$ \\ / $ \rm kJ \cdot mol^{-1} $} \\
		 				%	表格分行示例 不需要请删去
		 				\midrule
		 				1 & 1 & 1 & 3 & $-1.01316\times10^{6}$ \\
		 				2 & 2 & 1 & 7 & \\
		 				3 & 3 & 1 & 3 & \\
		 				4 & 4 & 1 & 6 & \\
		 				5 & 5 & 1 & 1 & \\
		 				\bottomrule
		 			\end{tabular}
		 		\end{table}
		 		\vbox{}
 	
	 		\subsubsection{实验内容2}
	 			实验内容2...
	 		\subsubsection{实验内容3}
	 			实验内容3...
	 		\subsubsection{实验内容4}
	 			实验内容4...
	 		\subsubsection{实验内容5}
	 			实验内容5...
	 			% 公式输入示例
	 			$$ E_{total}=E_{trans}+E_{rot}+E_{vib}+E_{elec} $$ 
	 	
	 			$$ 1\ \ {\rm a.u.}=2625.50 \ \ {\rm kJ \cdot mol^{-1}} $$
	 			
	 			$$\Delta E_{total}=E_{total}({\rm nap})-E_{total}{\rm (azu)=-1.4\times10^{2} \ \ kJ \cdot mol^{-1} }$$
	 	
	 			% 文献引用实例
	 		
	 	\subsection{实验结果及分析}
			\subsubsection{实验内容1}
	 			实验内容1...
 			
	\vbox{} 
 	
 	\section{讨论与结论}
		\subsection{实验讨论}
 			\subsubsection{实验讨论1}
 	 			实验讨论1...
 	 		\subsubsection{实验讨论2}
 	 			实验讨论2...
 	 		\subsubsection{实验讨论3}
 	 			实验讨论3...
 	 
 		 \subsection{实验结论}
 	 		实验结论...

\vbox{}  
% 参考文献
%\bibliographystyle{achemso}
%\bibliography{cite}

\end{document}